% Metódy inžinierskej práce

\documentclass[10pt,twoside,slovak,a4paper]{article}

\usepackage[slovak]{babel}
%\usepackage[T1]{fontenc}
\usepackage[IL2]{fontenc} % lepšia sadzba písmena Ľ než v T1
\usepackage[utf8]{inputenc}
\usepackage{graphicx}
\usepackage{url} % príkaz \url na formátovanie URL
\usepackage{hyperref} % odkazy v texte budú aktívne (pri niektorých triedach dokumentov spôsobuje posun textu)

\usepackage{cite}
%\usepackage{times}

\pagestyle{headings}

\title{Uplatnenie gamifikácie v športe\thanks{Semestrálny projekt v predmete Metódy inžinierskej práce, ak. rok 2022/23, vedenie: Ladislav Zemko}} % meno a priezvisko vyučujúceho na cvičeniach

\author{Rastislav Pašek\\[2pt]
	{\small Slovenská technická univerzita v Bratislave}\\
	{\small Fakulta informatiky a informačných technológií}\\
	{\small \texttt{xpasek@stuba.sk}}
	}

\date{\small 6. november 2022} % upravte



\begin{document}

\maketitle

\begin{abstract}
Princípy gamifikácie sa využívajú v širokom rozmedzí a majú množstvo kreácií. Cieľom tejto práce je zameranie sa na využitie týchto princípov v športe. Viaceré prieskumy poukazujú na nedostatok motivácie u ľudí ako hlavnú príčinu zhoršenia životosprávy. Možným riešením danej problematiky je aplikovanie princípu gamifikácie, ktoré by malo prispieť k zlepšeniu kvality života ľudí. Presnejšie ako motivácia na vykonávanie určitej fyzickej aktivity ako je beh a cvičenie, napríklad pomocou získania ocenenia za vykonanú aktivitu, ale aj zlepšenie životosprávy ako zdravé stravovanie a kvalitný spánok. Najúčinnejším spôsobom uplatnenia gamifikácie sa v tomto prípade javí použitie mobilnej aplikácie na to určenej.
\end{abstract}



\textbf {Kľúčové slová:} gamifikácia, motivácia, fitness, mobilná aplikácia, strava, kalorické tabuľky



\section{Úvod}

	V súčasnosti veľké množstvo ľudí nevykonáva žiadnu pravidelnú fyzickú aktivitu. Ovplyvnilo to viacero faktorov, pričom jedným z nich je napríklad Covid pandémia. Mnohí ľudia prestali športovať, chodiť na akcie rôzneho druhu, zvykli si nevychádzať zo svojich domovov. Všetko riešia online z pohodlia domova, či už objednanie jedla alebo pracovanie z domu. Na jednu stranu to uľahčuje život, ale zároveň to škodí v rôznych oblastiach, pretože ľudia sa stávajú lenivejšími a majú nedostatok motivácie na akúkoľvek fyzickú aktivitu. Taktiež konzumujú menej kvalitné jedlá, ktoré tvoria prevažne „fastfoody“. To negatívne vplýva na ich životosprávu a aj imunitu. 
	Jedným z riešením tohto problému je využitie princípu gamifikácie, či už prostredníctvom aplikácie v mobilnom telefóne alebo v smart-hodinkách. Táto aplikácia má za úlohu monitorovanie pohybu a určitých životných funkcií. Používatelia majú možnosť do nej zapisovať svoj každodenný príjem potravín a vody. Na základe týchto informácií im aplikácia vyhodnotí určité týždenné ciele, ktoré musia splniť. Za každý splnený cieľ používatelia dostanú nejakú odmenu, ktorá predstavuje istú motiváciu. 
	V tomto článku sa najskôr zameriavame na upresnenie pojmu gamifikácia a presných princípov, ktoré ďalej využívame v tejto práci. Následne upriamujeme pozornosť na konkrétny prostriedok využitia týchto princípov gamifikácie, tj. aplikáciu v mobilnom zariadení a jej používanie, a zároveň jej vysvetlenie na konkrétnej aplikácii.




\section{Gamifikácia} \label{Gamifikácia}

Gamifikácia je relatívne nová technika, ktorá sa snaží zvyšovať záujem pomocou využitia herných princípov a prostriedkov do neherných oblastí.\cite{Tóth:GamificationSA} Táto technika sa rozšírila do všetkých mediálnych odvetví. Najrozsiahlejšie využitie má vo firemnej oblasti a marketingu, kde ju môžeme rozdeliť na internú a externú. Interná sa zameriava na rozvoj firmy, prácu zamestnancov a ich výkonnosť, kde zapája herné prvky. Napríklad za vykonanie určitých pracovných úloh zamestnanci získavajú body, podľa ktorých sú zaradený do rebríčku výkonnosti, kde najvýkonnejší zamestnanci dostanú odmenu. Externá sa zameriava hlavne na zákazníkov a marketing, kde využíva kombinácie herných prostriedkov na oslovenie určitej skupiny ľudí, zlepšenie predajov. Napríklad program s vernostnými bodmi, kde za pravidelné nákupy v danom obchode zákazníci získavajú vernostné body, ktoré môžu využiť na zľavu nákupu, prípadne prémiových vecí, ktoré sa nedajú bežne kúpiť.  

\subsection {Princípy ktoré využívame}

Princípy, ktoré využívame v tejto práci sú kombináciou externej a internej časti, keďže sa snažíme osloviť čo najviac ľudí k využitiu aplikácie, ale zároveň sa snažíme zlepšovať ich životný štýl a motivovať ich na vykonávanie im prospešných aktivít. Využívame základné herné prostriedky ako sú poradový rebríček, „progress bar“, prémiový obchod, súťaže, odmeny. Z dlhodobého hľadiska by sme vďaka týmto prostriedkom mali udržať pozornosť používateľa natoľko, aby sa zlepšila jeho životospráva a motivácia k vykonávaniu fyzickej aktivity. Poradový rebríček je zoznam bodov jednotlivých používateľov zoradený vzostupne. Používateľ na čele rebríčka je ocenený určitou odmenou. Progress bar slúži na mapovanie našeho napredovania v plnení zadaných cieľov. Hlavnou funkciou prémiového obchodu je zamenenie získaných bodov za nejaký produkt...
\label{Gamifikácia}


\section{Prostriedok na využitie gamifikácie} \label{Využitie gamifikácie}
Túto aplikáciu princípov gamifikácie uplatňujeme prostredníctvom aplikácie do mobilného zariadenia. Hneď po spustení aplikácie si zvolíme cieľ, podľa toho si aplikácia vypýta údaje o nás ako je výška, váha, vek, zdravotné problémy, atď. Tieto informácie sú potrebné pre čo najvyššiu efektivitu tejto aplikácie, keďže podľa týchto informácií nám aplikácia vytvorí úlohy, ktoré musíme splniť. 

\subsection{Aplikácia Strava}
Aplikácia Strava má množstvo funkcií. Hlavnou z nich je zaznamenávanie športových aktivít. Vybrať si môžeme zo širokého množstva ponúkaných aktivít. Po zvolení jednej z nich stlačíme "Štart" a spustí sa nám časomiera a začne sa počítať prejdená vzdialenosť. Po skončení aktivity sa nás aplikácia opýta na doplňujúce otázky a detaily ohladom aktivity, napr. ako hodnotíme danú aktivitu, popísanie trasy a terénu, a taktiež či chceme našu aktivitu zdielať s ostatými používateľmi. Okrem iného môžeme pridať aj fotografie vyfotené počas aktivity a rovnako ich zdielať. Ďalšou z funkcií je možnosť zadať si svoje vlastné ciele, či už týždenné, mesačné alebo ročné. Za splnenie zadaného cieľa získame ocenenie v podobe odznaku konkrétnej kategórie. Aplikácia tiež ponúka možnosť zapojenia sa do úloh a výziev pre používateľov. Počas danej výzvy spĺňame aktivity, za ktoré získavame body do spoločného rebríčku, kde vidíme naše výsledky v porovnaní s ostatnými používateľmi. Na konci každej výzvy je zvolený víťaz podla najväčšieho počtu bodov.  
Všetky tieto funkcie prispievajú k spoločnému cieľu, ktorým je motivovať používaťeľov k vykonávaniu určitej aktivity. Porovnávanie našich výsledkov s ostatnými nás môže povzbudiť k zlepšeniu našich výkonov. Taktiež zdieľanie lokalít daných aktivít, poprípade priloženie fotografií, môže slúžiť ako inšpirácia dané miesta navštíviť a vyskúšať si aktivitu v tomto prostredí.

\subsection{Aplikácia Kalorické tabuľky}
Aplikácia Kalorické Tabuľky ponúka viacero fukncií, pričom tou hlavnou je zaznamenávanie kalórií potravín, ktoré konzumujeme. Na začiatku si zvolíme vlastné nutrienty a ciele aké chceme spĺňať. Ak nevieme hodnoty nutrientov, stačí zvoliť cieľ a aplikácia nám ich priradí automaticky, napr. chceme schudnúť, zadáme naše údaje ako je výška, váha, atď., a aplikácia nám následne nastaví hodnoty nutrientov na chudnutie. Do aplikácie zaznamenávame jedlá, ktoré konzumujeme, či už nascanovaním BAR kódu alebo vyhľadaním v zozname potravín. Ak sa dané jedlo nenachádza v zozname, máme možnosť ho do zoznamu pridať. Aplikácia nám taktiež ponúka široký sortiment receptov, vyhovujúcim našim potrebám. Jednou z funkcií je aj blog, kde nájdeme množstvo inšpirujúcich článkov, či už ohladom stravy, ale aj športových aktivít a zlepšenia samotnej žitovosprávy. Nachádza sa tu aj možnosť zaznamenávať našu aktivitu, avšak nie tak rozsiahlo ako v aplikácii Strava. Táto aplikácia je dostupná zadarmo, ale je tu aj možnosť zakúpiť si predplatenú verziu na rok, ktorá ponúka detailnejšie štatistiky a tipy na jedálničky.
Cieľom tejto aplikácie je sledovanie si a udržiavanie zdravého stravovania, ktoré je dôležité pre silnú imunitu a zdravie človeka. Prostredníctvom rád z blogu alebo z tipov, ktoré nám aplikácia ponúka, máme možnosť inšpirovať sa k príprave vlastnéhých jedál namiesto konzumácie "fast-foodov", a tým zároveň zlepšiť naše zdravie.






\section{Záver} \label{zaver} 
Využívanie týchto aplikácie sa preukázalo ako najúčinnejší spôsob využitia gamifikácie s cielom motivovať užívateľov.



%\acknowledgement{Ak niekomu chcete poďakovať\ldots}
\nocite{Bunturo:EGHFMA}
\nocite{Wijaya:DGCycling}
\bibliography{literatura}
\bibliographystyle{plain}
\end{document}
